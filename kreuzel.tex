% !TeX program=lualatex
\documentclass[parskip=half]{scrartcl}
\usepackage{luacode}
\usepackage{tabularx}
\usepackage{amssymb}
\pagestyle{empty} % Keine Seitennummer

% Parameter
\newcommand{\blattnummer}{3}
\newcommand{\gruppe}{} % Unterscheidung nach Übungsgruppen könnte einfach nachgerüstet werden. Ist aber vorerst nicht beabsichtigt.
\newcommand{\datei}{mathe.csv}

% Lua-Teil
\begin{luacode}
	require("kreuzellesen.lua")
	require("listeerzeugen.lua")
	kopfdaten, blattinfos, studenten = lese_datei("\datei")
\end{luacode}
\newcommand{\tabelleausgeben}{ \directlua{tex.print(print_table(kopfdaten, blattinfos, studenten, \blattnummer))} }
\newcommand{\kopfdaten}[1]{\directlua{tex.print(kopfdaten.#1)}}
\newcommand{\blattname}{\directlua{tex.print(blattinfos[\blattnummer].name)}}
\newcommand{\aufgabenBisher}{\directlua{tex.print(aufgaben_bisher( blattinfos, \blattnummer ))}}
\newcommand{\minimum}{\directlua{tex.print(kopfdaten.Minimum)}}


\begin{document}

	\section*{Gruppe 11, Montag 03.07.2017, 14:00 -- 16:00, \blattname}
	\if\aufgabenBisher0 \else Bisher gab es \aufgabenBisher{} Aufgaben. \fi Zur Zulassung zur Prüfung müssen Sie im gesamten Semester mindestens \minimum{}\,\% der Aufgaben ankreuzen.

	\tabelleausgeben

	Raum: PHY 5.1.03 \hfill \raggedleft \kopfdaten{MeinVorname} \kopfdaten{MeinNachname}, \kopfdaten{MeineEmail} %

\end{document}